% A place to put the high level information for the proposal and 
%

\section{Executive Summary}

\paragraph{Name of system / Purpose and scope: \\}
The BABAR/sPHENIX superconducting solenoid provides a central field up to 1.5T, a 2.84m warm bore, and a 3.5m coil length. The stored energy of the magnet is up to 27 MJ, and its operating temperature is 4.5K. This note provides an overview of the BaBar solenoid that will be re-purposed as the ECCE detector solenoid, documents the method by which the in-kind value of the solenoid is estimated. It documents simulations of the magnetic field and the estimated forces on the solenoid coils in the asymmetric ECCE configuration as well as studies of the shape of the magnetic field on the performance of the dual-radiator RICH (dRICH). 

\paragraph{Describe Technology Choice, with reference to Yellow Report: \\}
%The studies in the Yellow Report considered both a low-field (1.4-1.5T) and high field magnet (3.0T). 
The 1.5T magnet discussed here is the BABAR/sPHENIX magnet discussed in section 11.1 of the Yellow Report. Beyond the characteristics already described in the YR this note includes: a decision tree for selection of the the BABAR/sPHENIX magnet, evaluation of forces on the coils, impact of the magnetic field on PID performance in the forward endcap, and in particular on the dRICH performance.
%What is different, same, improved?}

\paragraph{Expected performance of system versus Yellow Report requirements: \\}
The BaBar solenoid provides 
%the 1.4T 
a nominal 1.5 magnetic field to the experiment. The performance of the tracking and momentum resolution 
%in 
including the lower magnetic field is described in the ECCE tracking systems analysis note. 

The field provided by the BaBar solenoid is sufficiently projective that it does not cause a significant smearing of the dRICH ring from the gaseous radiator, and therefore trim coils to shape the magnetic field are not required.  

%Could just be a table.}

\paragraph{Limitations if any for EIC physics: \\}
The nominal 1.5T field of the BaBar magnet taken by itself does not imply any limitations for EIC physics.  The performance of the different detector subsystems (tracking, etc.) are described in additional ECCE working group notes. 
% Include at least one key performance plot and/or links to key performance plots.
